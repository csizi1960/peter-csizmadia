\documentclass{article}[12pt]

\pagestyle{empty}

\begin{document}

\centerline{\bf\LARGE Curriculum Vitae}

\section*{Personal Data}

\begin{center}
\begin{tabular}{|ll|}\hline & \cr
Full Name: & Csizmadia, P{\'e}ter\cr & \cr
Nationality: & Hungarian\cr & \cr
Birth date and place: & October 9, 1972, Budapest, Hungary\cr & \cr
Marital status: & Single\cr &\cr
Mailing address: & KFKI Research Institute for Particle and\cr
		 & Nuclear Physics,\cr
& H-1525 Budapest POB 49, HUNGARY\cr & \cr
Tel.: & (+36)-1-392-2222/1613\cr
Fax: & (+36)-1-395-9151\cr
E-mail: & cspeter@rmki.kfki.hu\cr
& \cr\hline
\end{tabular}
\end{center}

\section*{Education}

\begin{itemize}
\item PhD of the E{\"o}tv{\"o}s University (Budapest, Hungary)
	in Nuclear Physics, 2003
\item Masters (Diploma) of the E{\"o}tv{\"o}s University (Budapest, Hungary)
	in Phy\-sics, 1996 
\end{itemize}

\section*{Employment}

\begin{itemize}
\item 1996-99: Research Assistant, MTA KFKI RMKI, Department of Theoretical
	Physics (Budapest, Hungary) 
\item 1999-: Research Fellow, MTA KFKI RMKI, Department of Theoretical
	Physics (Budapest, Hungary) 
\end{itemize}

\section*{Language and computer skills}

\begin{itemize}
\item English (good), German (basic), Hungarian (excellent)
\item C, C++, Java, Perl, HTML, JavaScript, Fortran (read only), Pascal, etc.
\item Unix, Linux, Windows systems
\end{itemize}

\newpage

\centerline{\bf\LARGE Publications}

\section*{Papers in Refereed Journals} %and in Preparation}

\begin{itemize}

\item P. Csizmadia, T. Cs{\"o}rg{\H o}, B. Luk{\'a}cs: {\it New analytic
      solutions of the non-relativistic hydrodynamical equations},
      Physics Letters {\bf B443} (1998) 21-25  [arXiv:nucl-th/9805006]

\item P. Csizmadia, P. L{\'e}vai, S. E. Vance, T. S. Bir{\'o}, M. Gyulassy,
      J. Zim{\'a}nyi: {\it Strange hyperon and antihyperon production from
      quark and string-rope matter}, Journal of Physics {\bf G25} (1999)
      321-330 [arXiv:hep-ph/9809456]

\item P. Csizmadia, P. L{\'e}vai: {\it $\phi$, $\Omega$ and $\rho$ production
      from deconfined matter in relativistic heavy ion collisions at CERN SPS},
      Physical Review {\bf C61} (2000) 031903 [arXiv:hep-ph/9909544]

\item P. L{\'e}vai, T. S. Bir{\'o}, P. Csizmadia, T. Cs{\"o}rg{\H o}, J.
      Zim{\'a}nyi: {\it The production of charm mesons from quark matter at
      CERN SPS and RHIC}, Journal of Physics {\bf G27} (2001) 703-706
      [arXiv:nucl-th/0011023]

\item S. Cheng, S. Pratt, P. Csizmadia, Y. Nara, D. Molnar, M. Gyulassy,
      S. E. Vance, B. Zhang: {\it The effect of finite-range interactions in
      classical transport theory}, Physical Review {\bf C65} (2002) 024901
      [arXiv:nucl-th/0107001]

\item P. Csizmadia and P. L{\'e}vai, {\it The MICOR hadronization model with
      final state interactions}, J. Phys. {\bf G28} (2002) 1997-2000

\item Y. Nara, S.E. Vance, P. Csizmadia, {\it A study of parton energy loss in
      Au+Au collisions at RHIC using transport theory},
      Physics Letters {\bf B531} (2002) 209-215 [arXiv:nucl-th/0109018]

\item P. Csizmadia and P. L{\'e}vai, {\it Energy dependence of transverse
      quark flow in heavy ion collisions},
      Acta Physica Hungarica {\bf A}22 (2005) 371-380 [arXiv:nucl-th/0407054]

\item P. Csizmadia, {\it Testing a new mesh refinement code in the evolution
      of a spherically symmetric Klein-Gordon field},
      International Journal of Modern Physics {\bf D}15 (2006) 107-119
      [arXiv:hep-th/0505036]

\item P. Csizmadia, {\it  Fourth order AMR and nonlinear dynamical systems
      in compactified space},
      Classical and Quantum Gravity {\bf 24} (2007) S369-S379.

\item G. Hamar, L.L. Zhu, P. Csizmadia, P. Levai, {\it The robustness of
      quasiparticle coalescence in quark matter},
      European Physical Journal Special Topics {\bf 155} (2008) 67-74.

\item G. Hamar, L.L. Zhu, P. Csizmadia, P. Levai, {\it Strange hadron yields
      and ratios in heavy ion collisions at RHIC energy},
      Journal of Physics {\bf G}35 (2008) 044067
      [arXiv:0710.4730].

\end{itemize}

\section*{Other publications}

\begin{itemize}
\item P. Csizmadia: {\it Coalescence in Quark Matter},
      TDK (Scientific Student Activity) thesis, 1995.

\item P. Csizmadia: {\it Hadronization of Quark Matter},
      Master (Diploma) Thesis, 1996.

\item P. Csizmadia, P. L{\'e}vai, J. Zim{\'a}nyi: {\it Pion Momentum
      Distribution from a Microscopical Hadronization Model}, Proc. of Int.
      Workshop on Gross Properties of Nuclei and Nuclear Excitations XXV,
      January 13-17, 1997, Hirschegg, Austria, (TH Darmstadt, 1997)
      Ed. by H. Feldmeier, p. 117.

\item P. Csizmadia, P. L{\'e}vai: {\it Hadron production in the MICOR model},
      Proc. of Int. Workshop on Understanding Deconfinement in QCD,
      ECT* Trento, Italy, March 1 - 13, 1999

\item P. Csizmadia, P. L{\'e}vai:
      {\it D and J/psi production from deconfined matter in
      relativistic heavy ion collisions}, arXiv:hep-ph/0008195,

\item G. Hamar, L.L. Zhu, P. Csizmadia, P. Levai: {\it Strange hadron yields
      and ratios in heavy ion collisions at RHIC energy}, arXiv:0710.4730.

\end{itemize}

\section*{Conference talks}

\begin{itemize}
\item {\it A microscopical model of quark matter hadronization},
      RHIC'98 Summer School, BNL, NY, USA, July 8-14, 1998
\item {\it Hadron production in the MICOR model},
      Int. Workshop on Understanding Deconfinement in QCD,
      ECT* Trento, Italy, March 1 - 13, 1999
\item {\it The MICOR hadronization model}, Parton'99 Workshop,
      Budapest, Hungary, May 4-7, 1999,
\item {\it MICOR hadronization model}, Subatech, Nantes, France, June 5-16, 2000
\item {\it The pion wind problem}, Hard Probe'2000 Workshop, BNL, NY, USA,
      August 1-21, 2000
\item {\it MICOR hadronization model}, Dense Matter Winter School, Schladming,
      Austria, March 3-10, 2001
\item {\it Hadronization and secondary interactions},
      Budapest, Hungary, March, 2001
\item {\it The MICOR hadronization model with final state interactions},
      Strange Quarks in Matter'2001, Frankfurt, Germany, September 22-28, 2001
\item {\it Energy dependence of the transverse flow in heavy ion collisions},
      Budapest'04 Workshop, Budapest, Hungary, 24-27 March, 2004
\item {\it Testing a High Precision Mesh Refinement Code in the Evolution of
      Massive Spherically Symmetric Fields},
      New Frontiers in Numerical Relativity Workshop,
      AEI, Golm, Germany, July 17-21, 2006
\end{itemize}

\section*{Posters}
\begin{itemize}
\item P. Csizmadia and I. R{\'a}cz,
      {\it On the cosmological relevance of oscillons},
      Frontiers in Numerical Gravitational Astrophysics Summer School,
      Erice, Italy, June 27-July 5, 2008
\end{itemize}

\section*{Conferences, Schools and Visits}

\begin{itemize}
\item Department of Physics, Bergen, Norway, October-November 1997
\item Workshop on Gross Properties of Nuclei and Nuclear Excitations XXV,
      Hirschegg, Austria,
      January 13-17, 1997
\item RHIC'98 Summer School, BNL, NY, USA, July 8-14, 1998
\item Int. Workshop on Understanding Deconfinement in QCD,
      ECT* Trento, Italy, March 1 - 13, 1999
\item Columbia University, New York, USA, January-February 2000
\item Subatech, Nantes, France, June 5-16, 2000
\item Hard Probe'2000 Workshop, BNL, NY, USA, August 1-21, 2000
\item Quark Matter'2000 Conference, BNL/Stony Brook, USA, January 15-20, 2001
\item Dense Matter Winter School, Schladming, Austria, March 3-10, 2001
\item Bergen Computational Physics Laboratory, Bergen, Norway, April 2001
\item Strange Quarks in Matter'2001 Conference, Frankfurt, Germany,
      September 22-28, 2001
\item Columbia University, New York, USA, October-November 2001
\item Quark Matter'2002 Conference, Nantes, France, July 18-24, 2002
\item Quark Matter'2004 Conference, Oakland, CA USA, 11-17 January, 2004
\item New Frontiers in Numerical Relativity Workshop,
      Albert Einstein Institute, Golm, Germany, July 17-21, 2006
\item 1st VESF School on Gravitational Waves, Cascina, Italy, May 22-26, 2006
\item Frontiers in Numerical Gravitational Astrophysics Summer School,
      Erice, Italy, June 27-July 5, 2008
\end{itemize}

\section*{Past and recent research}

%\begin{itemize}
%\item Heavy Ion Physics
%    \begin{itemize}
%    \item Hadronization
%    \item Hadronic rescattering
%    \item Hydrodinamics
%    \end{itemize}
%\item Computer code development
%    \begin{itemize}
%    \item Transport codes in C++
%    \item Monte Carlo simulations
%    \item Solving field equations in the conformal representation
%    \end{itemize}
%\end{itemize}
I started to work on quark matter hadronization in 1995. I have developed a
model called MICOR that is based on the assumption that constituent quark
matter is created in high energy heavy ion collisions. The aim was to explain
the experimental momentum spectra of hadrons in CERN SPS Pb+Pb and RHIC Au+Au
collisions by a hadronization process similar to coalescence.
%However, the MICOR final state is a resonance gas, so a realistic simulation
%of a heavy ion collision should contain an additional component. To get stable
%particles from this initial state, I apply a cascade program.

In 2000, I began working on a simplified model of rescattering processes, the
pion wind model, using the GCP generic cascade program. In the same year, we
started the development of a new general cascade code, called PSYCHE, with S.
E. Vance and Y. Nara.  In 2001, the RHIC Transport Theory Collaboration was
formed, with our program in the center. The latest version of the program,
called Gromit is even more general, not just the interactions, particles and
initial conditions but also the cascade algorithm is definable.
It can be used to perform a ``full'' secondary interaction simulation starting
from arbitrary initial state. In my investigations, I used my hadronization
model, MICOR, to generate the initial states.

In 2005, I created a high precision adaptive mesh refinement (AMR) code,
called GridRipper. Initially, it was
applied to simulate the time evolution of physical fields in flat space-time,
such as the Klein-Gordon equation and the Yang-Mills-Higgs system. In general,
the code is able to solve hyperbolic systems of partial differential equations
using the chosen integration scheme (like RK2, RK4 or ICN). The initial
condition is either defined by arbitrary formulae, generated by program
code or produced as the numerical solution of an ordinary differential
equation. In 2008, we found a new numerical framework with I. R{\'a}cz,
which made it possible to follow the time evolution of the Einstein equations
even after the appearance of trapped surfaces, until a singularity is reached.
Based on this finding, I applied GridRipper in problems related to cosmic
inflation and the gravitational collapse of fluids and scalar fields in
spherical symmetry, leading to the birth of baby universes.

\end{document}
