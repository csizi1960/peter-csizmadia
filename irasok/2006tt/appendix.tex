\documentclass{article}[12pt]

\usepackage[magyar]{babel}
\usepackage[utf8]{inputenc}

\usepackage[colorlinks=true,
filecolor=blue,linkcolor=blue,citecolor=blue,urlcolor=blue
]{hyperref}

\begin{document}

\section{Besugárzás által gyorsított bomlás torzítása}

Tegyük fel, hogy a mintában kezdetben $N_X(0)$ mennyiségben jelenlévő $X$
izotóp bomlásából $Y$ izotóp keletkezik $T$ felezési idővel, azonban egy
besugárzás hatására mindjárt az idők kezdetén $q N_X(0)$ mennyiségű izotóp
elbomlik. Így $t$ idő elteltével $X$ izotóp mennyisége
\begin{equation}
N_X(t)\,\ =\,\ 2^{-t/T}(1-q)N_X(0),
\end{equation}
$Y$ mennyisége pedig
\begin{eqnarray}
N_Y(t) &=& N_Y(0) + N_X(0) - N_X(t)\\
&=& N_Y(0) + (2^{t - T\log_2(1-q)\over T} - 1)N_X(t).
\end{eqnarray}
A minta $t$ életkorára egyenesillesztéssel kapott eredményben így a sugárzás
$-T\log_2(1-q)$ eltérést okoz, amely láthatóan arányos a felezési idővel. Az
eltérés tehát nagy mértékben függ az anyagtól, amelyet a kormeghatározáshoz
használunk.

\end{document}
